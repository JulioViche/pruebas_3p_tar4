\documentclass[12pt,a4paper]{article}

% ============================================================
% Paquetes
% ============================================================
\usepackage[utf8]{inputenc}
\usepackage[spanish]{babel}
\usepackage[T1]{fontenc}
\usepackage{lmodern}
\usepackage{geometry}
\usepackage{graphicx}
\usepackage{xcolor}
\usepackage{hyperref}
\usepackage{titlesec}
\usepackage{fancyhdr}
\usepackage{enumitem}
\usepackage{parskip}
\usepackage{booktabs}
\usepackage{listings}
\usepackage{tcolorbox}
\usepackage{float}

% ============================================================
% Configuración de página
% ============================================================
\geometry{
    left=2.5cm,
    right=2.5cm,
    top=2.5cm,
    bottom=2.5cm
}

% ============================================================
% Colores personalizados
% ============================================================
\definecolor{primaryblue}{RGB}{0, 82, 155}
\definecolor{accentblue}{RGB}{0, 123, 193}
\definecolor{codegray}{RGB}{245, 245, 245}
\definecolor{codegreen}{RGB}{40, 167, 69}

% ============================================================
% Configuración de hyperref
% ============================================================
\hypersetup{
    colorlinks=true,
    linkcolor=primaryblue,
    urlcolor=accentblue,
    citecolor=primaryblue,
    pdfauthor={Julio Viche},
    pdftitle={Informe de Práctica: Proyecto Node.js con CI},
    pdfsubject={Pruebas de Software}
}

% ============================================================
% Configuración de títulos
% ============================================================
\titleformat{\section}
    {\normalfont\Large\bfseries\color{primaryblue}}
    {\thesection}{1em}{}[\titlerule]

\titleformat{\subsection}
    {\normalfont\large\bfseries\color{accentblue}}
    {\thesubsection}{1em}{}

% ============================================================
% Configuración de encabezados y pies de página
% ============================================================
\pagestyle{fancy}
\fancyhf{}
\fancyhead[L]{\footnotesize\textcolor{gray}{Pruebas de Software}}
\fancyhead[R]{\footnotesize\textcolor{gray}{Julio Viche}}
\fancyfoot[C]{\thepage}
\renewcommand{\headrulewidth}{0.4pt}
\renewcommand{\footrulewidth}{0pt}

% ============================================================
% Configuración de listings para código
% ============================================================
\lstset{
    backgroundcolor=\color{codegray},
    basicstyle=\ttfamily\small,
    breaklines=true,
    frame=single,
    rulecolor=\color{gray!30},
    xleftmargin=0.5cm,
    xrightmargin=0.5cm
}

% ============================================================
% Ruta de imágenes
% ============================================================
\graphicspath{{img/}}

% ============================================================
% Documento
% ============================================================
\begin{document}

% ============================================================
% Portada
% ============================================================
\begin{titlepage}
    \centering
    \vspace*{2cm}
    
    {\Huge\bfseries\color{primaryblue} Informe de Práctica\par}
    \vspace{0.5cm}
    {\LARGE\color{accentblue} Proyecto Node.js con CI\par}
    
    \vspace{2cm}
    
    \begin{tcolorbox}[
        colback=primaryblue!5,
        colframe=primaryblue,
        width=0.85\textwidth,
        arc=3mm,
        boxrule=1pt
    ]
        \centering
        \renewcommand{\arraystretch}{1.5}
        \begin{tabular}{rl}
            \textbf{Estudiante:} & Julio Viche \\
            \textbf{Materia:} & Pruebas de Software \\
            \textbf{Docente:} & Ing. Enrique Calvopiña \\
            \textbf{Fecha:} & 22 de enero de 2026 \\
        \end{tabular}
    \end{tcolorbox}
    
    \vspace{1.5cm}
    
    {\large\textbf{Tema:}\par}
    {\large Desarrollo de proyecto Node.js con pruebas unitarias y CI\par}
    
    \vfill
    
\end{titlepage}

% ============================================================
% Tabla de contenidos
% ============================================================
\tableofcontents
\newpage

% ============================================================
% Introducción
% ============================================================
\section{Introducción}

Este informe documenta el desarrollo de un proyecto Node.js siguiendo la guía proporcionada por el docente. Se implementó una estructura básica, pruebas unitarias, configuración de ESLint y un flujo de integración continua (CI) con GitHub Actions. Se incluyen evidencias de cada paso realizado.

% ============================================================
% Objetivos
% ============================================================
\section{Objetivos}

\begin{itemize}[leftmargin=*, itemsep=0.3em]
    \item Implementar la estructura base de un proyecto Node.js.
    \item Configurar dependencias y herramientas de calidad de código.
    \item Desarrollar pruebas unitarias y asegurar su ejecución automática mediante CI.
    \item Documentar el proceso y resultados obtenidos.
\end{itemize}

% ============================================================
% Desarrollo
% ============================================================
\section{Desarrollo}

\subsection{Estructura del Proyecto}

Se creó la siguiente estructura de carpetas y archivos:

\begin{figure}[H]
    \centering
    \includegraphics[width=0.3\textwidth]{part1_step1.png}
    \caption{Estructura del repositorio}
\end{figure}

\subsection{Instalación de Dependencias}

\begin{itemize}[leftmargin=*, itemsep=0.3em]
    \item Inicialización del proyecto: \lstinline|npm init -y|
    \item Instalación de Express: \lstinline|npm install express|
    \item Instalación de Jest y ESLint: \lstinline|npm install --save-dev jest eslint|
\end{itemize}

\begin{figure}[H]
    \centering
    \includegraphics[width=0.9\textwidth]{part1_step2.png}
    \caption{Instalación de dependencias}
\end{figure}

\subsection{Creación de Archivos Base}

\textbf{index.js:} Servidor Express con endpoint básico.

\begin{figure}[H]
    \centering
    \includegraphics[width=0.7\textwidth]{part2_step1.png}
    \caption{Archivo index.js}
\end{figure}

\textbf{sum.js:} Función de suma exportable.

\begin{figure}[H]
    \centering
    \includegraphics[width=0.8\textwidth]{part2_step2.png}
    \caption{Archivo sum.js}
\end{figure}

\textbf{sum.test.js:} Prueba unitaria para suma.

\begin{figure}[H]
    \centering
    \includegraphics[width=0.7\textwidth]{part2_step3.png}
    \caption{Archivo sum.test.js}
\end{figure}

\textbf{math.js:} Funciones factorial y fibonacci.

\begin{figure}[H]
    \centering
    \includegraphics[width=0.9\textwidth]{activities_math.png}
    \caption{Archivo math.js}
\end{figure}

\textbf{math.test.js:} Pruebas unitarias para factorial y fibonacci.

\begin{figure}[H]
    \centering
    \includegraphics[width=0.7\textwidth]{activities_math_test.png}
    \caption{Archivo math.test.js}
\end{figure}

\subsection{Configuración de Scripts y ESLint}

Scripts en \texttt{package.json} para start, test y lint.

\begin{figure}[H]
    \centering
    \includegraphics[width=0.6\textwidth]{part2_step4.png}
    \caption{Archivo package.json con scripts configurados}
\end{figure}

Configuración de ESLint en formato flat config en \texttt{eslint.config.js}.

\begin{figure}[H]
    \centering
    \includegraphics[width=0.8\textwidth]{part2_step5.png}
    \caption{Archivo eslint.config.js}
\end{figure}

Se utilizó \texttt{.gitignore} para \texttt{node\_modules} y archivos temporales.

\subsection{Configuración de Git y CI}

Inicialización de repositorio Git y primer commit.

\begin{figure}[H]
    \centering
    \includegraphics[width=0.8\textwidth]{part3_step2.png}
    \caption{Repositorio en GitHub}
\end{figure}

Configuración de GitHub Actions en \texttt{.github/workflows/ci.yml} para ejecutar lint y pruebas en cada push o pull request.

\begin{figure}[H]
    \centering
    \includegraphics[width=0.5\textwidth]{part3_step3.png}
    \caption{Archivo ci.yml de GitHub Actions}
\end{figure}

Evidencia del funcionamiento de GitHub Actions.

\begin{figure}[H]
    \centering
    \includegraphics[width=0.8\textwidth]{part3_step4.png}
    \caption{GitHub Actions en funcionamiento}
\end{figure}

\subsection{Actividades con GitHub Actions}

Al pushear \texttt{math.js} y \texttt{math.test.js}:

\begin{figure}[H]
    \centering
    \includegraphics[width=0.8\textwidth]{activities_step1.png}
    \caption{Actividad 1: Push de archivos math}
\end{figure}

Se modificó \texttt{sum.js} agregando un fallo a propósito.

\begin{figure}[H]
    \centering
    \includegraphics[width=0.7\textwidth]{activities_intentional_error.png}
    \caption{Error intencional en sum.js}
\end{figure}

Al pushear el cambio en \texttt{sum.js}:

\begin{figure}[H]
    \centering
    \includegraphics[width=0.8\textwidth]{activities_step2.png}
    \caption{Actividad 2: CI detecta el error}
\end{figure}

Al corregir y pushear otra vez:

\begin{figure}[H]
    \centering
    \includegraphics[width=0.8\textwidth]{activities_correction.png}
    \caption{Corrección del error y CI exitoso}
\end{figure}

% ============================================================
% Resultados Obtenidos
% ============================================================
\section{Resultados Obtenidos}

\begin{itemize}[leftmargin=*, itemsep=0.3em]
    \item Se demostró cómo se ejecutan pruebas de manera automática en \texttt{sum.js} y \texttt{math.js} usando \texttt{sum.test.js} y \texttt{math.test.js} respectivamente en GitHub Actions.
    \item Se demostró que se detectan errores de manera efectiva.
\end{itemize}

% ============================================================
% Conclusiones
% ============================================================
\section{Conclusiones}

\begin{enumerate}[leftmargin=*, itemsep=0.5em]
    \item La integración de pruebas unitarias y flujos de CI en proyectos Node.js permite detectar errores de manera temprana y automática, mejorando la calidad del software.
    
    \item Se presentó un inconveniente con la configuración de ESLint, ya que la versión utilizada requiere el nuevo formato Flat Config (\texttt{export default}), lo que obligó a adaptar la configuración en el commit \href{https://github.com/JulioViche/pruebas_3p_tar4/commit/bf74fd05633e93f785d238400bf569ad52cb98ec}{bf74fd0} para que ESLint funcionara correctamente en la versión 9+.
    
    \item Se documentó y evidenció cada paso para facilitar la comprensión y replicabilidad del proceso.
\end{enumerate}

% ============================================================
% Recomendaciones
% ============================================================
\section{Recomendaciones}

\begin{enumerate}[leftmargin=*, itemsep=0.5em]
    \item Verificar siempre la versión de las herramientas instaladas y consultar la documentación oficial para evitar problemas de compatibilidad, como ocurrió con ESLint.
    
    \item Mantener el código y las configuraciones actualizadas según las mejores prácticas recomendadas por la comunidad.
    
    \item Comentar el código y documentar el proceso para facilitar el mantenimiento y la colaboración en equipo.
\end{enumerate}

% ============================================================
% Anexos
% ============================================================
\section{Anexos}

\begin{itemize}[leftmargin=*]
    \item \href{https://github.com/JulioViche/pruebas_3p_tar4}{Repositorio público de GitHub}
\end{itemize}

\end{document}
